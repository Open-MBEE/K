
\section{Translating K to SMT-LIB}

In this section we illustrate the translation from K to the SMT-
LIB input language. SMT-LIB \cite{smt-lib} is the standard 
``satisfiability modulo theories library'' for SMT solvers. The 
standard is used by numerous SMT solvers, allowing comparison 
between systems (for example in competitions). 
In addition, it allows systems generating SMT-LIB 
formulas to target any SMT solver processing this standard. In our 
case we  use the Z3 SMT solver to process the generated formulas, 
but we anticipate to target other solvers as well in future work.

\subsection{The Source K Model}

\begin{figure}
\centering
\begin{tabular}{c}
\hline \\
\lstinputlisting{examples/spacecraft.k} \\ \\
\hline
\end{tabular}
\caption{A simple \Klang{} model of a spacecraft}
\label{fig:spacecraftSmt}
\end{figure}

The translator currently covers a subset of the K language
corresponding to the model shown in Figure 
\ref{fig:spacecraftSmt}. The example illustrates the features
of K that have been used by engineers at JPL until the time of 
writing. The emphasis of these models is on {\em structure}
of artifacts and {\em scheduling} of events. The model models a 
spacecraft. The class \code{Object} is meant to represent entities 
that have weight. Instruments, and its radio sub-classes, as well
as the the spacecraft itself are objects, inheriting from this 
class. The class \code{Instrument} defines a \code{power} level 
consumed. Requirements in the form of Boolean constraints are 
imposed on \code{power} and \code{weight}. The 
\code{SpaceCraft} class makes instances of instruments,
defines a combined sum \code{instrumentsWeight} and a
constraint on it with additional requirements. The elements of
the model discussed above are so-called {\em structural} elements,
what one would normally see in a class diagram.

The spacecraft in addition contains a system manager, representing
the software on board. For the purpose of illustration, the system manager is defined as a small {\em scheduler} of three events: 
a \code{bootUp} event, re-booting the flight software computer, a \code{initMem} event, initializing the computer memory, and a 
\code{takePicture} event, taking a picture. An event is a constant
of the \code{Event} class, which defines an event as having a start time and and end time appearing after the start time. In addition, the \code{Event} class defines a function \code{after},
which as argument takes another event `\code{e}', 
and returns true if the event (this) occurs after `\code{e}'.
The \code{after} function is part of Allen logic 
\cite{allen-logic-84} used by the planning and scheduling community. 
%
Finally, the model contains an instance \code{ShRaan} of type
\code{SpaceCraft}.

Given the spacecraft model, the general proof-theoretic
problem we want an answer to is whether our classes are logically consistent. That is, whether the constraints of each class are consistent (do not evaluate to {\em false} such as for example is
the case with: `$x < 0 \wedge x > 0$'). From a semantics point of view, it means that for each class there exists at least one instance (object) of that class that satisfies the constraints. 
The specific satisfiability problem that perhaps interests a
user most is whether there is an instance of the \code{SpaceCraft} class, which satisfies all the constraints of that class and the classes it refers to.

\subsection{The Translation to SMT-LIB}

\lstset{language=SMT}

For the 44 line K model in Figure \ref{fig:spacecraftSmt},
the translator generates 333 lines of uncommented SMT-LIB code (additional comments are generated to make the output easier for humans to read). ...

\hline

\begin{center}
\begin{tabular}{c}
\begin{lstlisting}
(define-sort Ref () Int)
\end{lstlisting}
\end{tabular}
\end{center}

\hline

\begin{center}
\begin{tabular}{c}
\begin{lstlisting}
(declare-datatypes () ((Object 
  (mk-Object (weight Int)))))
  
(declare-datatypes () ((SpaceCraft 
  (mk-SpaceCraft (weight Int)
                 (instrumentsWeight Real)
                 (radio Ref)
                 (camera Ref)
                 (software Ref)))))
...
\end{lstlisting}
\end{tabular}
\end{center}

\hline

\begin{center}
\begin{tabular}{c}
\begin{lstlisting}
(declare-datatypes () ((Any
  (lift-Object (sel-Object Object))
  (lift-SpaceCraft (sel-SpaceCraft SpaceCraft))
  ...
  (lift-SystemManager (sel-SystemManager SystemManager))
  null))
)
\end{lstlisting}
\end{tabular}
\end{center}

\hline

\begin{center}
\begin{tabular}{c}
\begin{lstlisting}
(declare-const heap (Array Ref Any))

(define-fun deref ((ref Ref)) Any
  (select heap ref)
)
\end{lstlisting}
\end{tabular}
\end{center}

\hline

\begin{center}
\begin{tabular}{c}
\begin{lstlisting}
(define-fun deref-is-SpaceCraft ((this Ref)) Bool
  (is-lift-SpaceCraft (deref this))
)

(define-fun deref-SpaceCraft ((this Ref)) SpaceCraft
  (sel-SpaceCraft (deref this))
)
\end{lstlisting}
\end{tabular}
\end{center}

\hline

\begin{center}
\begin{tabular}{c}
\begin{lstlisting}
(define-fun deref-isa-Instrument ((this Ref)) Bool
  (or
    (deref-is-Instrument this)
    (deref-is-SimpleRadio this)
    (deref-is-SmartRadio this)
  )
)
\end{lstlisting}
\end{tabular}
\end{center}

\hline

\begin{center}
\begin{tabular}{c}
\begin{lstlisting}
(define-fun Instrument!weight ((this Ref)) Int
  (weight (deref-Instrument this))
)

(define-fun Instrument.weight ((this Ref)) Int
  (ite (deref-is-Instrument this) 
    (weight (deref-Instrument this))
    (ite (deref-is-SimpleRadio this) 
      (weight (deref-SimpleRadio this))
      (weight (deref-SmartRadio this)))))
\end{lstlisting}
\end{tabular}
\end{center}

\hline

\begin{center}
\begin{tabular}{c}
\begin{lstlisting}
(define-fun Event.after ((this Ref)(e Ref)) Bool
  (>= (Event.start this)  (Event.end e))
)

(define-fun Event!after ((this Ref)(e Ref)) Bool
  (>= (Event!start this)  (Event.end e))
)
\end{lstlisting}
\end{tabular}
\end{center}

\hline

\begin{center}
\begin{tabular}{c}
\begin{lstlisting}
(define-fun SystemManager.inv ((this Ref)) Bool
  (and
    (deref-isa-Event (SystemManager!bootUp this))
    (deref-isa-Event (SystemManager!initMem this))
    (deref-isa-Event (SystemManager!takePicture this))
    (and 
      (Event.after 
        (SystemManager!takePicture this)  
        (SystemManager!initMem this)) 
      (Event.after 
        (SystemManager!takePicture this)  
        (SystemManager!bootUp this))
    )
  )
)
\end{lstlisting}
\end{tabular}
\end{center}

\hline

\begin{center}
\begin{tabular}{c}
\begin{lstlisting}
(assert (forall ((this Ref))
  (=> (deref-is-SystemManager this) (SystemManager.inv this))
))
\end{lstlisting}
\end{tabular}
\end{center}

\hline

\begin{center}
\begin{tabular}{c}
\begin{lstlisting}
(assert (exists ((instanceOfInstrument Ref)) 
  (deref-is-Instrument instanceOfInstrument)))
...
\end{lstlisting}
\end{tabular}
\end{center}

\hline

\begin{figure}
\VerbatimInput{examples/spacecraftOutput.k}
\caption{Output of the K toolchain for the spacecraft example.}
\label{fig:shapes}
\end{figure}

\lstset{language=K}
