\section{\Klang{} in PRACTICE}
\label{sec:usage}

Currently, \Klang{} is used to analyze models created for the NASA
Europa Clipper Mission Concept. Figure~\ref{fig:k} gives an overview
of the usage scenarios for the \Klang{} language and tool chain.

The typical scenario involves modelers (of the Europa Clipper mission
concept) creating SysML diagrams in a tool such as MagicDraw and
saving them to a central model repository. This database of models is
accessible via a REST API. The input to the REST API is a unique
identifier for a node (typically a \sysml{} package) in the model, and
the result is a list of all the nodes that are part of the package
specified in the input. The result is provided as an array of JSON
objects, where each object contains information such as name, type,
owner, etc. Typically, the types of objects are classes, constraints,
expressions, and member properties. The \Klang{} tool chain takes this
input and converts each node in the list of nodes to a corresponding
\Klang{} AST object. Since the list of nodes received from the REST
API is unordered and unstructured, we perform multiple passes on the
list of nodes. The first pass is performed to create the list of
classes in the model, followed by passes to populate properties and
constraints in each class. Once the \Klang{} model has been
constructed, the \Klang{} tool chain proceeds normally with type
checking and SMT analysis. Currently this scenario is based on a {\em
  pull} methodology where a modeler has to initiate the \Klang{} based
translation and analysis. In the future, we plan on automating this
effort and have it be executed on a regular cadence with results made
available through the model database to a web application.

\begin{figure*}
  \centering
  \includegraphics[scale=0.49]{K.png}
  \caption{\Klang{} in practice.}
  \label{fig:k}
\end{figure*}



\begin{comment}
\begin{figure*}
\centering
\includegraphics[scale=0.32]{kweb.png}
\caption{\Klang{} editor in the browser.}
\label{fig:k}
\end{figure*}
\end{comment}

A second common scenario for using \Klang{} is via the web
browser~\cite{kwebsite}. We have created a simple HTML based \Klang{}
code editor along with the functionality to invoke the \Klang{}
tool chain from the web browser. This page is used for purposes of
teaching, learning, exploring, and prototyping with \Klang{}. The web
page also provides a tutorial, documentation, and examples.

Finally, the \Klang{} tool chain is also available as a binary
download for all major operating systems. Users may download the
binaries and invoke the tool chain from the command line. We expect
that certain models of Europa Clipper mission concept are created and
analyzed directly as \Klang{} models. Currently, there are two such
salient examples where \Klang{} was used to create and analyze
requirements. The first model is a series of scheduling constraints,
which after modeling in \Klang{}, were successfully analyzed in less
than 20 seconds. The results of the analysis also discovered a
scheduling problem, that was later successfully confirmed by a
significantly more cumbersome manual analysis. The second model
contains a series of high level constraints that are analyzed using
\Klang{} for satisfiability. This model solves in around 30
seconds. Due to JPL's information release restrictions, such details
cannot be shared at this time.

