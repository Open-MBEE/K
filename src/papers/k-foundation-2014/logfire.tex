
\section{The \logfire{} Rule Engine}
\label{sec:logfire}

As already mentioned, \logfire{} is a \scala{} API for writing
rule-based programs in a manner that has the appearance of a DSL.  It
was originally created as a study of how the
\rete{} algorithm could be used for runtime verification purposes,
where the main goal is to check event traces against formalized
specifications, and emit verdicts in a Boolean domain, stating whether
the event stream satisfies the specification or not. In the following,
we shall first illustrate the originally intended application, and in
the subsequent sub-section we shall illustrate its use for abstraction,
which is the topic of this paper. We then suggest that a more
convenient solution is desirable for this objective.

\subsection{\logfire{} used for Verification}

Consider a system that emits two kinds of events: $E_1(clk \rightarrow
t_1)$ and $E_2(clk \rightarrow t_2)$, each being a named record (names are
$E_1$ and $E_2$) with a field $clk$ that is mapped to a time stamp $t_i$
indicating the time when these events were generated. Suppose we want
to enforce that $E_2$ can only occur after $E_1$, and furthermore, if
$E_2$ occurs, it has to occur within 5 seconds of the occurrence of
$E_1$.  This property is shown in Figure
\ref{fig:logfire-verifier}. The main component of \logfire{} is the
{\bf trait}\footnote{A {\bf trait} in \scala{} is a module concept
closely related to the notion of an {\em abstract class}, as for
example found in \java.} \name{Monitor}, which any user-defined
monitor must extend to get access to the constants and methods
provided by the rule DSL. The {\em events} $E_1$ and $E_2$ are
short-lived instantaneous observations about the system being
monitored, those submitted to the monitor.  In contrast, {\em facts},
in this case $E_1Seen$, are long-lived pieces of information stored in
the fact memory of the rule system, generated and deleted explicitly
by the rules.  In the monitor above the fact $E_1Seen(t_1)$ is used to
represent the fact that the event $E_1(clk \rightarrow t_1)$ has been
seen.  The monitor contains three rules, named $v_1$, $v_2$ and
$v_3$. Each rule has the form:

\[
  \textit{name}\ \verb+--+\ \textit{condition}_1\ \& \ldots \&\ \textit{condition}_n\ \longmapsto\  \textit{action}
\]

\begin{figure}[t]
\begin{small}
\begin{framed}
\sscala
\begin{lstlisting}
class Verifier extends Monitor {
  "v1" -- 'E1('clk -> 't1) |-> insert('E1Seen('t1))

  "v2" -- 'E2('clk -> 't2) & not('E1Seen('t1)) |-> fail()
  
  "v3" -- 'E2('clk -> 't2) & 'E1Seen('t1) |-> {
    if('t2-'t1 > 5000) fail()
  }
}
\end{lstlisting}
\end{framed}
\end{small}
\caption{A \logfire{} verifier}
\label{fig:logfire-verifier}
\end{figure}

Event and fact names, as well as parameter names are values of the
\scala{} type \name{Symbol}, which contains quoted identifiers.  The
need for representing user-defined names as symbols is a consequence
of the fact that \logfire{} is a deep embedding (we don't use
\scala{}'s names). 
%and that names are not first-class citizens in
%\scala{} as they are for example in \lisp.
%
Events and facts can have arguments specified in one of two ways:
using {\em positional notation} or {\em using map notation}\label{page:mapnotation}. 
Positional notation means
just listing arguments as a list of patterns (identifiers or literals). In our
example facts are represented using positional notation. 
The positional notation is convenient if events/facts carry few arguments. 
%
Map notation means considering the events/facts as being maps from field names to values. 
In our example event patterns are shown using map notation, assuming each event has a time stamp named \name{'clk}. When using map notation only fields relevant for the rule need be mentioned. An action is any \scala{} statement, that specifically for example can add or delete facts, or call failure methods.

The rules are to be read as follows. Rule $v_1$ states that when an $E_1$ event
is observed, a fact, $E_1Seen$ is created to record this. Rule $v_2$ states that an error is generated if an $E_2$ event is observed, but no $E_1$ event has been observed before that. Finally, rule $v_3$ states that in the case an $E_1$ event and subsequently an $E_2$ event is observed, the time difference must be within $5$
seconds. 
%
A monitor can be applied as shown in Figure
\ref{fig:logfire-verifier-application}, which also shows an example of
an error trace produced.  Each entry in the error trace shows the
number of the event, the event, the fact that it causes to be
generated, and the rule that triggers. In this case the $5$ second requirement
is violated.

\begin{figure}[htb]
\begin{small}
\begin{framed}
\sscala
\begin{lstlisting}
object ApplyMonitor {
  def main(args: Array[String]) {
    val m = new Verifier
    m.addMapEvent('E1)('clk -> 1023)
    m.addMapEvent('E3)('clk -> 3239)
    m.addMapEvent('E2)('clk -> 7008)
  }
}
...

*** error:

[1] 'E1('clk->1023) --> 'E1Seen(1023)
        rule: "v1" -- 'E1('clk->'t1) |-> {...}

[3] 'E2('clk->7008) --> 'Fail("ERROR")
        rule: "v3" -- 'E2('clk->'t2) & 'E1Seen('t1) |-> {...}
\end{lstlisting}
\end{framed}
\end{small}
\caption{Applying a \logfire{} verifier}
\label{fig:logfire-verifier-application}
\end{figure}


\subsection{\logfire{} used for Abstraction}

In this sub-section we shall illustrate how \logfire{} may be used to
model the hierarchical behaviors of interest in our application.  We
consider a scenario with a top-level behavior (denoted $alpha$) that
consists of an inner behavior (denoted $beta$) in parallel with a single
event $E_3$.  The behavior $beta$ in turn consists of two events
$E_1$ and $E_2$ that must occur in that order.  Denoting the three
atomic events as $E_1(clk \rightarrow t_1)$, $E_2(clk \rightarrow
t_2)$, and $E_3(clk \rightarrow t_3)$, we want to record two facts:
that $E_2$ occurs after $E_1$ is to be recorded as an occurrence of
$beta(t_1,t_2)$, and that $E_3$ occurs either before or after (that is: in parallel with) $beta(t_1,t_2)$ is to be recorded as an occurrence of
$alpha(t_1,t_2,t_3)$.  The resulting monitor is shown in
Figure~\ref{fig:logfire-abstracter}.

\begin{figure}
\begin{small}
\begin{framed}
\sscala
\begin{lstlisting}
class Abstracter extends Monitor {
  "a1" -- 'E1('clk -> 't1) |-> insert('E1Seen('t1))

  "a2" -- 'E1Seen('t1) & 'E2('clk -> 't2) |-> { 
    remove('E1Seen); 
    insert('beta('t1, 't2)) 
  }
  
  "a3" -- 'E3('clk -> 't3) |-> insert('E3Seen('t3))
  
  "a4" -- 'beta('t1, 't2) & 'E3Seen('t3) |-> { 
    remove('E3Seen); 
    insert('alpha('t1, 't2, 't3)) 
  }
}
\end{lstlisting}\end{framed}
\end{small}
\caption{A \logfire{} abstracter}
\label{fig:logfire-abstracter}
\end{figure}

The monitor contains four rules. The first rule, $a_1$, records when an $E_1(clk \rightarrow t_1)$
event is seen. Rule $r_2$ records a $beta(t_1,t_2)$ fact when an 
$E_2(clk \rightarrow t_2)$ event is seen after an $E_1(clk \rightarrow t_1)$ event. It also removes the intermediate
event recording that $E_1$ was seen, in order to not clutter the set of facts generated. Rule $a_3$ records when an $E_3(clk \rightarrow t_3)$ 
event is seen, and finally rule $a_4$ creates the $alpha(t_1,t_2,t_3)$ fact.
When applying the abstracter to the same event sequence as shown in 
Figure \ref{fig:logfire-verifier-application}, instead of an error trace, we obtain
a set of generated facts, as shown in Figure \ref{fig:logfire-abstracter-application}.

\begin{figure}[htb]
\begin{small}
\begin{framed}
\sscala
\begin{lstlisting}
object ApplyMonitor {
  def main(args: Array[String]) {
    val m = new Abstracter
    m.addMapEvent('E1)('clk -> 1023)
    m.addMapEvent('E3)('clk -> 3239)
    m.addMapEvent('E2)('clk -> 7008)
  }
}
...

--- facts: -------
'beta(1023,7008)
'alpha(1023,7008,3239)
--------------
\end{lstlisting}
\end{framed}
\end{small}
\caption{Applying a \logfire{} abstracter}
\label{fig:logfire-abstracter-application}
\end{figure}

The main observation to be made, about this specification, as well as
the verifier in Figure \ref{fig:logfire-verifier}, is that it is
inconvenient that we have to add (and delete) intermediate facts such as
$E_1Seen$ and $E_3Seen$ explicitly, which makes these rules cumbersome
to write and maintain.  To avoid this problem, in the next section, we
introduce notation that allows hierarchical events to be described
more directly, in a form similar to the way one writes regular
expressions, but with support for conjunctive composition and event parameters.

