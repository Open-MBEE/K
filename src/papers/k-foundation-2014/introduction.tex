
\section{Introduction}
\label{sec:introduction}

% motivate problem first
% - complex missions with limited visibility
% - increasing data volumes downlinked
% - tight deadlines on operational turnaround

% current approach
% - ad-hoc scripts written in scripting languages
% - summaries presented through interactive, visual interfaces
% - for approach to be useful, must capture understanding of system design
% - must be able to represent hierarchical relationships between system elements

% previous work
% - focused on runtime checking of properties
% - too limiting for our purposes
% - no support for hierarchical, sequential relationships among behaviors

% contributions
% - provide a declarative language for capturing system design understanding
% - extensible (using internal DSL)
% - bridge gap between fully formal specs and ad-hoc scripts

One of the key challenges in operating remote spacecraft is that
ground operators must rely on limited telemetry visible on the ground
in order to assess the health and operational status of the
spacecraft.  Such telemetry typically consists of a log of system
events and sensor measurements (such as battery voltage or probe
temperature) which, for the purposes of this paper, may be viewed as a
sequence of timestamped records with named fields.  Because this
telemetry comprises essentially {\em all} the knowledge that ground
operators have about a given spacecraft, processing this telemetry in
a timely manner is of utmost importance to any mission.  However, as
spacecraft have become more autonomous and capable, and improvements
in radio performance have resulted in greater downlink bandwidth, the
resulting volume and complexity of the telemetry requires more
automated processing tools so that any potential problems are
diagnosed quickly and accurately.  Unfortunately, currently such tools
are developed by ground operators in an ad-hoc manner, typically using
libraries developed by various subsystem teams that mine the telemetry
to infer summaries that are of interest to that subsystem.  These
summaries are typically presented to ground operators using various
visualization interfaces.  While these tools have been overall quite
effective, and the domain knowledge encoded in these libraries
has led to many problems being identified
early, the current approach of relying on ad-hoc scripts
also makes the resulting tools fragile, hard for new team members to
understand, and difficult to maintain.  The maintainability issue is
especially important for long-running missions that are expected to
last many years.

To address this problem, this paper presents a declarative notation
for expressing domain-specific knowledge about telemetry structure.
In our formalism, the behavior of spacecraft subsystems may be
expressed in terms of {\em behaviors}, in a language that resembles
regular expressions, but with support for conjunction and data
arguments to nonterminals.  A key feature of our notation is that
behaviors may be nested, since in our experience, most subsystems are
usually viewed as a set of (possibly interrelated) hierarchical
behaviors, and often viewed using visualization interfaces that allow
behaviors of interest to be explored interactively.  We demonstrate
our approach in practice by showing how it is being applied to
telemetry received from the Curiosity rover~\cite{msl} currently on
Mars.  However, although applied to spacecraft operation, the
techniques are fully general, and can be used for analysis of any form
of event logs produced by a software system.

There has been much previous work in processing telemetry event logs
in the field of {\em runtime verification} (RV), typically checking logs
against user provided specifications, often expressed in some form of
temporal logic.  Such analysis may take place pre-deployment, as the
system is being developed, or post-deployment, during operation.
Orthogonally, the monitoring may be done online, processing telemetry
on-board during execution, or offline by analyzing logs.  We
are concerned with post-deployment offline trace analysis.  Most
previous work, however, focuses on checking if a given event log
satisfies a given specification or not (sometimes extending the
Boolean domain with extra values to $3$ or $4$-valued logics,
indicating grade of satisfaction).  In our experience, coming up with
formally verifiable properties is difficult in practice, especially
for complex missions where design requirements were not in a formal
notation to begin with.  Thus our focus is more on providing a
framework for performing {\em log comprehension}.  This form of log
comprehension can often be useful in identifying problems not easily
formalized, but can also serve as a stepping stone to eventually
writing (traditional) formal properties that may be checked for
satisfaction.

To provide flexibility in expressing varied subsystem models, we have
implemented our notation as an internal DSL (Domain-Specific
Language), essentially an API, in the \scala{}
programming language. \scala{} offers language constructs that makes
definition of such APIs have the appearance of DSLs.  Specifically we
use \scala's implicit functions to define concrete syntax, and case
classes to define abstract syntax. The resulting DSL is largely a
so-called {\em deep embedding}, in contrast to a {\em shallow}
embedding. In a deep embedding a particular program in the DSL is
completely defined by an abstract syntax tree, which can be processed
as an internal data structure. In contrast, in a shallow embedding
host language constructs are made part of the DSL.  
A deep embedding makes it easier to analyze DSL programs.
As we shall see,
however, we do allow the DSL to contain arbitrary \scala{} code in
limited positions, hence our approach is a mix of deep and shallow
embedding.

We implement our DSL using the rule-based \logfire{}
system~\cite{havelund-logfire-sttt14}, which is itself an internal
\scala{} DSL.  Rule-based systems, which have been extensively studied
within the artificial intelligence (AI) community, allow formulation of
rules of the form:

\[
  condition_1, \ldots, condition_n \Rightarrow action
\]

\noindent
The state of a rule-system can abstractly be considered as consisting
of a set of {\em facts}, referred to as the {\em fact memory}, where a
fact is a mapping from field names to values.  A condition in a rule's
left-hand side can check for the presence or absence of a particular
fact. A left-hand side matching against the fact memory usually
requires unification of variables occurring in conditions. In case all
conditions on a rule's left-hand side match (become true), the
right-hand side action is executed, which can be any \scala{} code,
including adding and deleting facts, or generating error
messages. \logfire{} is an implementation of the \rete{} algorithm
\cite{forgy-rete-82} used in many AI rule systems.

The rule formalism, although very natural and expressive, turns out to
be slightly verbose for writing log properties. The core problem
can be illustrated by an example. Assume that one wants to monitor
that the events $E_1$ and $E_2$ occur in that order. A rule system
would have to explicitly create an intermediate fact $E_{1}Seen$
representing the fact that $E_1$ has occurred. The issue is similar to
that of state machines where all states must be explicitly
defined. Regular expressions and temporal logics provide a solution to
this problem. We here show a regular expression-like formalism which
(i) makes this more convenient, and (ii) which allows for abstraction
as discussed above.  The DSL we present is defined as patterns that
are translated to rules, in a similar manner as discussed in
\cite{havelund-logfire-sttt14}.

The paper is organized as follows.
%
Section \ref{sec:related-work} outlines related work.
%
Section \ref{sec:logfire} introduces briefly the rule-based system \logfire{},
and outlines the inconveniences in using this solution for this problem.
%
Section \ref{sec:translation} introduces the new DSL and its translation to rules.
%
Section \ref{sec:application} presents the application to a spacecraft scenario,
illustrating visualization of event abstractions.
%
Finally, Section \ref{sec:conclusion} concludes the paper.
