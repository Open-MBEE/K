

\section{OCL}
\label{sec:ocl}

The disambiguating approach taken in Eclipse OCL is to parse an unambiguous larger language that unifies all the ambiguities. Subsequent semantic validation distinguishes between the ambiguities and diagnoses expressions from the larger language that are not valid OCL expressions.

From a technical point of view this makes the grammar simpler and more regular, and the implementation more modular and configurable by the library model.

From a user’s point of view, slightly wrong expressions may be syntactically valid and so semantic validation may produce a more helpful diagnostic. However completion assist may offer illegal expressions from the larger language.

{\bf Common between OCL and K Predicate Logic:}

\begin{itemize}
  \item ...
\end{itemize}

{\bf OCL in addition has (which K does not):}

\begin{itemize}
  \item tiples with named fields. K uses classes for this. OCL has this probably because
  OCL was created isolated from the structural part.
  \item supports the use of types as values in an expression.
\end{itemize}




