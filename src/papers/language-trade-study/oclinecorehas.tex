
\section{OCLInEcore Advantages}
\label{sec:oclinecorepros}

\begin{itemize}
  \item Packages can be arbitrarily nested.
  \item Datatypes (not sure what they are used for). Maybe for lifting primitive
    types (such as Int and Bool) to classifiers. Maybe to define subtypes
    of other types (one can provide invariant constraints). K possibly has some/all 
    of this capability. We need to know more about datatypes.
  \item Enumeration types.
  \item A class may be declared abstract or an interface.
  \item Static attributes and references (not supported by Ecore 
    itself).
  \item Multiplicities include optional (?).
  \item The defaults for multiplicity lower and upper bound and for 
    ordered and unique follow the UML specification and so 
    corresponds to a single element Set that is [1] \{unique,!
    ordered\}. Note that UML defaults differ from the Ecore 
    defaults which correspond to an optional element OrderedSet, 
    that is [?] \{ordered,unique\}.
  \item A property can an an optional opposite name, as in
  {\em cars\#garage}. Here {\em garage} is the opposite name.
  \item Supports side-effects, for example by supporting the notion
    of initial value of a property, or derived value, which one 
    would have to use functions for in K.
  \item Operations may be static.
  \item Operation parameters can be untyped.
  \item Operations can throw exceptions
  \item Invariants can be declared callable.
  \item A class invariant can be associated with a second
    expression, which is evaluated and returned to the user
    if the class invariant itself fails.
\end{itemize}